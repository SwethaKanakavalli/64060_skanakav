% Options for packages loaded elsewhere
\PassOptionsToPackage{unicode}{hyperref}
\PassOptionsToPackage{hyphens}{url}
%
\documentclass[
]{article}
\title{Assignment4}
\author{Swetha}
\date{3/14/2022}

\usepackage{amsmath,amssymb}
\usepackage{lmodern}
\usepackage{iftex}
\ifPDFTeX
  \usepackage[T1]{fontenc}
  \usepackage[utf8]{inputenc}
  \usepackage{textcomp} % provide euro and other symbols
\else % if luatex or xetex
  \usepackage{unicode-math}
  \defaultfontfeatures{Scale=MatchLowercase}
  \defaultfontfeatures[\rmfamily]{Ligatures=TeX,Scale=1}
\fi
% Use upquote if available, for straight quotes in verbatim environments
\IfFileExists{upquote.sty}{\usepackage{upquote}}{}
\IfFileExists{microtype.sty}{% use microtype if available
  \usepackage[]{microtype}
  \UseMicrotypeSet[protrusion]{basicmath} % disable protrusion for tt fonts
}{}
\makeatletter
\@ifundefined{KOMAClassName}{% if non-KOMA class
  \IfFileExists{parskip.sty}{%
    \usepackage{parskip}
  }{% else
    \setlength{\parindent}{0pt}
    \setlength{\parskip}{6pt plus 2pt minus 1pt}}
}{% if KOMA class
  \KOMAoptions{parskip=half}}
\makeatother
\usepackage{xcolor}
\IfFileExists{xurl.sty}{\usepackage{xurl}}{} % add URL line breaks if available
\IfFileExists{bookmark.sty}{\usepackage{bookmark}}{\usepackage{hyperref}}
\hypersetup{
  pdftitle={Assignment4},
  pdfauthor={Swetha},
  hidelinks,
  pdfcreator={LaTeX via pandoc}}
\urlstyle{same} % disable monospaced font for URLs
\usepackage[margin=1in]{geometry}
\usepackage{color}
\usepackage{fancyvrb}
\newcommand{\VerbBar}{|}
\newcommand{\VERB}{\Verb[commandchars=\\\{\}]}
\DefineVerbatimEnvironment{Highlighting}{Verbatim}{commandchars=\\\{\}}
% Add ',fontsize=\small' for more characters per line
\usepackage{framed}
\definecolor{shadecolor}{RGB}{248,248,248}
\newenvironment{Shaded}{\begin{snugshade}}{\end{snugshade}}
\newcommand{\AlertTok}[1]{\textcolor[rgb]{0.94,0.16,0.16}{#1}}
\newcommand{\AnnotationTok}[1]{\textcolor[rgb]{0.56,0.35,0.01}{\textbf{\textit{#1}}}}
\newcommand{\AttributeTok}[1]{\textcolor[rgb]{0.77,0.63,0.00}{#1}}
\newcommand{\BaseNTok}[1]{\textcolor[rgb]{0.00,0.00,0.81}{#1}}
\newcommand{\BuiltInTok}[1]{#1}
\newcommand{\CharTok}[1]{\textcolor[rgb]{0.31,0.60,0.02}{#1}}
\newcommand{\CommentTok}[1]{\textcolor[rgb]{0.56,0.35,0.01}{\textit{#1}}}
\newcommand{\CommentVarTok}[1]{\textcolor[rgb]{0.56,0.35,0.01}{\textbf{\textit{#1}}}}
\newcommand{\ConstantTok}[1]{\textcolor[rgb]{0.00,0.00,0.00}{#1}}
\newcommand{\ControlFlowTok}[1]{\textcolor[rgb]{0.13,0.29,0.53}{\textbf{#1}}}
\newcommand{\DataTypeTok}[1]{\textcolor[rgb]{0.13,0.29,0.53}{#1}}
\newcommand{\DecValTok}[1]{\textcolor[rgb]{0.00,0.00,0.81}{#1}}
\newcommand{\DocumentationTok}[1]{\textcolor[rgb]{0.56,0.35,0.01}{\textbf{\textit{#1}}}}
\newcommand{\ErrorTok}[1]{\textcolor[rgb]{0.64,0.00,0.00}{\textbf{#1}}}
\newcommand{\ExtensionTok}[1]{#1}
\newcommand{\FloatTok}[1]{\textcolor[rgb]{0.00,0.00,0.81}{#1}}
\newcommand{\FunctionTok}[1]{\textcolor[rgb]{0.00,0.00,0.00}{#1}}
\newcommand{\ImportTok}[1]{#1}
\newcommand{\InformationTok}[1]{\textcolor[rgb]{0.56,0.35,0.01}{\textbf{\textit{#1}}}}
\newcommand{\KeywordTok}[1]{\textcolor[rgb]{0.13,0.29,0.53}{\textbf{#1}}}
\newcommand{\NormalTok}[1]{#1}
\newcommand{\OperatorTok}[1]{\textcolor[rgb]{0.81,0.36,0.00}{\textbf{#1}}}
\newcommand{\OtherTok}[1]{\textcolor[rgb]{0.56,0.35,0.01}{#1}}
\newcommand{\PreprocessorTok}[1]{\textcolor[rgb]{0.56,0.35,0.01}{\textit{#1}}}
\newcommand{\RegionMarkerTok}[1]{#1}
\newcommand{\SpecialCharTok}[1]{\textcolor[rgb]{0.00,0.00,0.00}{#1}}
\newcommand{\SpecialStringTok}[1]{\textcolor[rgb]{0.31,0.60,0.02}{#1}}
\newcommand{\StringTok}[1]{\textcolor[rgb]{0.31,0.60,0.02}{#1}}
\newcommand{\VariableTok}[1]{\textcolor[rgb]{0.00,0.00,0.00}{#1}}
\newcommand{\VerbatimStringTok}[1]{\textcolor[rgb]{0.31,0.60,0.02}{#1}}
\newcommand{\WarningTok}[1]{\textcolor[rgb]{0.56,0.35,0.01}{\textbf{\textit{#1}}}}
\usepackage{graphicx}
\makeatletter
\def\maxwidth{\ifdim\Gin@nat@width>\linewidth\linewidth\else\Gin@nat@width\fi}
\def\maxheight{\ifdim\Gin@nat@height>\textheight\textheight\else\Gin@nat@height\fi}
\makeatother
% Scale images if necessary, so that they will not overflow the page
% margins by default, and it is still possible to overwrite the defaults
% using explicit options in \includegraphics[width, height, ...]{}
\setkeys{Gin}{width=\maxwidth,height=\maxheight,keepaspectratio}
% Set default figure placement to htbp
\makeatletter
\def\fps@figure{htbp}
\makeatother
\setlength{\emergencystretch}{3em} % prevent overfull lines
\providecommand{\tightlist}{%
  \setlength{\itemsep}{0pt}\setlength{\parskip}{0pt}}
\setcounter{secnumdepth}{-\maxdimen} % remove section numbering
\ifLuaTeX
  \usepackage{selnolig}  % disable illegal ligatures
\fi

\begin{document}
\maketitle

a.Use only the numerical variables (1 to 9) to cluster the 21 firms.
Justify the various choices made in conducting the cluster analysis,
such as weights for different variables, the specific clustering
algorithm(s) used, the number of clusters formed, and so on.

\begin{Shaded}
\begin{Highlighting}[]
\CommentTok{\#Loading the Required packages}
\FunctionTok{library}\NormalTok{(flexclust)}
\end{Highlighting}
\end{Shaded}

\begin{verbatim}
## Warning: package 'flexclust' was built under R version 4.1.3
\end{verbatim}

\begin{verbatim}
## Loading required package: grid
\end{verbatim}

\begin{verbatim}
## Loading required package: lattice
\end{verbatim}

\begin{verbatim}
## Loading required package: modeltools
\end{verbatim}

\begin{verbatim}
## Loading required package: stats4
\end{verbatim}

\begin{Shaded}
\begin{Highlighting}[]
\FunctionTok{library}\NormalTok{(cluster)}
\end{Highlighting}
\end{Shaded}

\begin{verbatim}
## Warning: package 'cluster' was built under R version 4.1.3
\end{verbatim}

\begin{Shaded}
\begin{Highlighting}[]
\FunctionTok{library}\NormalTok{(tidyverse)  }
\end{Highlighting}
\end{Shaded}

\begin{verbatim}
## Warning: package 'tidyverse' was built under R version 4.1.3
\end{verbatim}

\begin{verbatim}
## -- Attaching packages --------------------------------------- tidyverse 1.3.1 --
\end{verbatim}

\begin{verbatim}
## v ggplot2 3.3.5     v purrr   0.3.4
## v tibble  3.1.6     v dplyr   1.0.8
## v tidyr   1.2.0     v stringr 1.4.0
## v readr   2.1.2     v forcats 0.5.1
\end{verbatim}

\begin{verbatim}
## Warning: package 'readr' was built under R version 4.1.3
\end{verbatim}

\begin{verbatim}
## Warning: package 'forcats' was built under R version 4.1.3
\end{verbatim}

\begin{verbatim}
## -- Conflicts ------------------------------------------ tidyverse_conflicts() --
## x dplyr::filter() masks stats::filter()
## x dplyr::lag()    masks stats::lag()
\end{verbatim}

\begin{Shaded}
\begin{Highlighting}[]
\FunctionTok{library}\NormalTok{(factoextra) }
\end{Highlighting}
\end{Shaded}

\begin{verbatim}
## Warning: package 'factoextra' was built under R version 4.1.3
\end{verbatim}

\begin{verbatim}
## Welcome! Want to learn more? See two factoextra-related books at https://goo.gl/ve3WBa
\end{verbatim}

\begin{Shaded}
\begin{Highlighting}[]
\FunctionTok{library}\NormalTok{(FactoMineR)}
\end{Highlighting}
\end{Shaded}

\begin{verbatim}
## Warning: package 'FactoMineR' was built under R version 4.1.3
\end{verbatim}

\begin{Shaded}
\begin{Highlighting}[]
\FunctionTok{library}\NormalTok{(ggcorrplot)}
\end{Highlighting}
\end{Shaded}

\begin{verbatim}
## Warning: package 'ggcorrplot' was built under R version 4.1.3
\end{verbatim}

\begin{Shaded}
\begin{Highlighting}[]
\CommentTok{\#loading the data}
\FunctionTok{getwd}\NormalTok{()}
\end{Highlighting}
\end{Shaded}

\begin{verbatim}
## [1] "C:/Users/mercy/OneDrive/Desktop/FML/Assignment4"
\end{verbatim}

\begin{Shaded}
\begin{Highlighting}[]
\FunctionTok{setwd}\NormalTok{(}\StringTok{"C:/Users/mercy/OneDrive/Desktop/FML/Assignment4"}\NormalTok{)}
\NormalTok{Info}\OtherTok{\textless{}{-}} \FunctionTok{read.csv}\NormalTok{(}\StringTok{"Pharmaceuticals.csv"}\NormalTok{)}
\CommentTok{\# I am selecting columns from 3 to 11 and storing the data in variable Info1}
\NormalTok{Info1 }\OtherTok{\textless{}{-}}\NormalTok{ Info[}\DecValTok{3}\SpecialCharTok{:}\DecValTok{11}\NormalTok{]}
\CommentTok{\# Using head function to display the first 6 rows of data}
\FunctionTok{head}\NormalTok{(Info1)}
\end{Highlighting}
\end{Shaded}

\begin{verbatim}
##   Market_Cap Beta PE_Ratio  ROE  ROA Asset_Turnover Leverage Rev_Growth
## 1      68.44 0.32     24.7 26.4 11.8            0.7     0.42       7.54
## 2       7.58 0.41     82.5 12.9  5.5            0.9     0.60       9.16
## 3       6.30 0.46     20.7 14.9  7.8            0.9     0.27       7.05
## 4      67.63 0.52     21.5 27.4 15.4            0.9     0.00      15.00
## 5      47.16 0.32     20.1 21.8  7.5            0.6     0.34      26.81
## 6      16.90 1.11     27.9  3.9  1.4            0.6     0.00      -3.17
##   Net_Profit_Margin
## 1              16.1
## 2               5.5
## 3              11.2
## 4              18.0
## 5              12.9
## 6               2.6
\end{verbatim}

\begin{Shaded}
\begin{Highlighting}[]
\FunctionTok{summary}\NormalTok{(Info1)}
\end{Highlighting}
\end{Shaded}

\begin{verbatim}
##    Market_Cap          Beta           PE_Ratio          ROE      
##  Min.   :  0.41   Min.   :0.1800   Min.   : 3.60   Min.   : 3.9  
##  1st Qu.:  6.30   1st Qu.:0.3500   1st Qu.:18.90   1st Qu.:14.9  
##  Median : 48.19   Median :0.4600   Median :21.50   Median :22.6  
##  Mean   : 57.65   Mean   :0.5257   Mean   :25.46   Mean   :25.8  
##  3rd Qu.: 73.84   3rd Qu.:0.6500   3rd Qu.:27.90   3rd Qu.:31.0  
##  Max.   :199.47   Max.   :1.1100   Max.   :82.50   Max.   :62.9  
##       ROA        Asset_Turnover    Leverage        Rev_Growth   
##  Min.   : 1.40   Min.   :0.3    Min.   :0.0000   Min.   :-3.17  
##  1st Qu.: 5.70   1st Qu.:0.6    1st Qu.:0.1600   1st Qu.: 6.38  
##  Median :11.20   Median :0.6    Median :0.3400   Median : 9.37  
##  Mean   :10.51   Mean   :0.7    Mean   :0.5857   Mean   :13.37  
##  3rd Qu.:15.00   3rd Qu.:0.9    3rd Qu.:0.6000   3rd Qu.:21.87  
##  Max.   :20.30   Max.   :1.1    Max.   :3.5100   Max.   :34.21  
##  Net_Profit_Margin
##  Min.   : 2.6     
##  1st Qu.:11.2     
##  Median :16.1     
##  Mean   :15.7     
##  3rd Qu.:21.1     
##  Max.   :25.5
\end{verbatim}

\hypertarget{the-variables-are-measured-in-different-weights-throughout-the-rows-we-will-scale-the-data-in-info1-and-save-the-scaled-data-in-the-info2-dataframe.-i-am-calculating-the-distance-between-the-rows-of-data-and-visualizing-the-distance-matrix-using-get_dist-and-fviz_dist-functions-which-are-available-in-factoextra-package.}{%
\section{The variables are measured in different weights throughout the
rows, we will scale the data in Info1 and save the scaled data in the
Info2 dataframe. I am Calculating the distance between the rows of data
and visualizing the distance matrix using get\_dist and fviz\_dist
functions which are available in factoextra
package.}\label{the-variables-are-measured-in-different-weights-throughout-the-rows-we-will-scale-the-data-in-info1-and-save-the-scaled-data-in-the-info2-dataframe.-i-am-calculating-the-distance-between-the-rows-of-data-and-visualizing-the-distance-matrix-using-get_dist-and-fviz_dist-functions-which-are-available-in-factoextra-package.}}

\begin{Shaded}
\begin{Highlighting}[]
\NormalTok{Info2 }\OtherTok{\textless{}{-}} \FunctionTok{scale}\NormalTok{(Info1)}
\FunctionTok{row.names}\NormalTok{(Info2) }\OtherTok{\textless{}{-}}\NormalTok{ Info[,}\DecValTok{1}\NormalTok{]}
\NormalTok{distance }\OtherTok{\textless{}{-}} \FunctionTok{get\_dist}\NormalTok{(Info2)}
\FunctionTok{fviz\_dist}\NormalTok{(distance)}
\end{Highlighting}
\end{Shaded}

\includegraphics{Assignment4_files/figure-latex/unnamed-chunk-2-1.pdf}
\# Now I am creating the correlation Matrix and I am printing to check
the correlation among major variables

\begin{Shaded}
\begin{Highlighting}[]
\NormalTok{corr }\OtherTok{\textless{}{-}} \FunctionTok{cor}\NormalTok{(Info2)}
\FunctionTok{ggcorrplot}\NormalTok{(corr, }\AttributeTok{outline.color =} \StringTok{"grey50"}\NormalTok{, }\AttributeTok{lab =} \ConstantTok{TRUE}\NormalTok{, }\AttributeTok{hc.order =} \ConstantTok{TRUE}\NormalTok{, }\AttributeTok{type =} \StringTok{"full"}\NormalTok{) }
\end{Highlighting}
\end{Shaded}

\includegraphics{Assignment4_files/figure-latex/unnamed-chunk-3-1.pdf}
\#The Correlation Matrix reveals that ROA, ROE, Net Profit Margin, and
Market Cap is high. By using Principal Component Analysis I am finding
out weightage of major variables in the data set. Here I am assuming
best number of cluster is 5.

\begin{Shaded}
\begin{Highlighting}[]
\NormalTok{pca }\OtherTok{\textless{}{-}} \FunctionTok{PCA}\NormalTok{(Info2)}
\end{Highlighting}
\end{Shaded}

\includegraphics{Assignment4_files/figure-latex/unnamed-chunk-4-1.pdf}
\includegraphics{Assignment4_files/figure-latex/unnamed-chunk-4-2.pdf}

\begin{Shaded}
\begin{Highlighting}[]
\NormalTok{var }\OtherTok{\textless{}{-}} \FunctionTok{get\_pca\_var}\NormalTok{(pca)}
\FunctionTok{fviz\_pca\_var}\NormalTok{(pca, }\AttributeTok{col.var=}\StringTok{"contrib"}\NormalTok{,}
             \AttributeTok{gradient.cols =} \FunctionTok{c}\NormalTok{(}\StringTok{"grey"}\NormalTok{,}\StringTok{"yellow"}\NormalTok{,}\StringTok{"purple"}\NormalTok{,}\StringTok{"red"}\NormalTok{,}\StringTok{"blue"}\NormalTok{),}\AttributeTok{ggrepel =} \ConstantTok{TRUE}\NormalTok{ ) }\SpecialCharTok{+} \FunctionTok{labs}\NormalTok{( }\AttributeTok{title =} \StringTok{"PCA Variable Variance"}\NormalTok{)}
\end{Highlighting}
\end{Shaded}

\includegraphics{Assignment4_files/figure-latex/unnamed-chunk-4-3.pdf}
We may deduce from PCA Variable Variance that ROA,ROE, Net Profit
Margin, Market Cap, and Asset Turnover contribute over 61\% to the two
PCA components/dimensions (Variables)and I am using elbow method to find
optimal number of customers.

\begin{Shaded}
\begin{Highlighting}[]
\FunctionTok{set.seed}\NormalTok{(}\DecValTok{10}\NormalTok{)}
\NormalTok{wss }\OtherTok{\textless{}{-}} \FunctionTok{vector}\NormalTok{()}
\ControlFlowTok{for}\NormalTok{(i }\ControlFlowTok{in} \DecValTok{1}\SpecialCharTok{:}\DecValTok{10}\NormalTok{) wss[i] }\OtherTok{\textless{}{-}} \FunctionTok{sum}\NormalTok{(}\FunctionTok{kmeans}\NormalTok{(Info2,i)}\SpecialCharTok{$}\NormalTok{withinss)}
\FunctionTok{plot}\NormalTok{(}\DecValTok{1}\SpecialCharTok{:}\DecValTok{10}\NormalTok{, wss , }\AttributeTok{type =} \StringTok{"b"}\NormalTok{ , }\AttributeTok{main =} \FunctionTok{paste}\NormalTok{(}\StringTok{\textquotesingle{}Cluster of Companies\textquotesingle{}}\NormalTok{) , }\AttributeTok{xlab =} \StringTok{"Number of Clusters"}\NormalTok{, }\AttributeTok{ylab=}\StringTok{"wss"}\NormalTok{)}
\end{Highlighting}
\end{Shaded}

\includegraphics{Assignment4_files/figure-latex/unnamed-chunk-5-1.pdf}

\begin{Shaded}
\begin{Highlighting}[]
\NormalTok{wss}
\end{Highlighting}
\end{Shaded}

\begin{verbatim}
##  [1] 180.00000 118.56934  95.99420  79.21748  65.61035  52.67476  47.66961
##  [8]  41.12605  31.81763  31.57252
\end{verbatim}

I got the same number as assumed. Optimal cluster is at 5 . \#\#
Silhouette Method Finding best number of clusters.

\begin{Shaded}
\begin{Highlighting}[]
\FunctionTok{fviz\_nbclust}\NormalTok{(Info2, kmeans, }\AttributeTok{method =} \StringTok{"silhouette"}\NormalTok{)}
\end{Highlighting}
\end{Shaded}

\includegraphics{Assignment4_files/figure-latex/unnamed-chunk-6-1.pdf}
Here also the idealnumber of clusters is 5. Using k-means algorithm to
cluster with 5.

\begin{Shaded}
\begin{Highlighting}[]
\FunctionTok{set.seed}\NormalTok{(}\DecValTok{1}\NormalTok{)}
\NormalTok{k5 }\OtherTok{\textless{}{-}} \FunctionTok{kmeans}\NormalTok{(Info2, }\AttributeTok{centers =} \DecValTok{5}\NormalTok{, }\AttributeTok{nstart =} \DecValTok{25}\NormalTok{) }\CommentTok{\# k = 5, number of restarts = 25}
\NormalTok{k5}\SpecialCharTok{$}\NormalTok{centers }
\end{Highlighting}
\end{Shaded}

\begin{verbatim}
##    Market_Cap       Beta    PE_Ratio        ROE        ROA Asset_Turnover
## 1 -0.76022489  0.2796041 -0.47742380 -0.7438022 -0.8107428     -1.2684804
## 2 -0.43925134 -0.4701800  2.70002464 -0.8349525 -0.9234951      0.2306328
## 3 -0.03142211 -0.4360989 -0.31724852  0.1950459  0.4083915      0.1729746
## 4 -0.87051511  1.3409869 -0.05284434 -0.6184015 -1.1928478     -0.4612656
## 5  1.69558112 -0.1780563 -0.19845823  1.2349879  1.3503431      1.1531640
##      Leverage Rev_Growth Net_Profit_Margin
## 1  0.06308085  1.5180158      -0.006893899
## 2 -0.14170336 -0.1168459      -1.416514761
## 3 -0.27449312 -0.7041516       0.556954446
## 4  1.36644699 -0.6912914      -1.320000179
## 5 -0.46807818  0.4671788       0.591242521
\end{verbatim}

\begin{Shaded}
\begin{Highlighting}[]
\NormalTok{k5}\SpecialCharTok{$}\NormalTok{size  }
\end{Highlighting}
\end{Shaded}

\begin{verbatim}
## [1] 4 2 8 3 4
\end{verbatim}

\begin{Shaded}
\begin{Highlighting}[]
\FunctionTok{fviz\_cluster}\NormalTok{(k5, }\AttributeTok{data =}\NormalTok{ Info2) }
\end{Highlighting}
\end{Shaded}

\includegraphics{Assignment4_files/figure-latex/unnamed-chunk-7-1.pdf}
\#kmeans clustering, using Manhattan Distance

\begin{Shaded}
\begin{Highlighting}[]
\FunctionTok{set.seed}\NormalTok{(}\DecValTok{1}\NormalTok{)}
\NormalTok{k51 }\OtherTok{=} \FunctionTok{kcca}\NormalTok{(Info2, }\AttributeTok{k=}\DecValTok{5}\NormalTok{, }\FunctionTok{kccaFamily}\NormalTok{(}\StringTok{"kmedians"}\NormalTok{))}
\NormalTok{k51}
\end{Highlighting}
\end{Shaded}

\begin{verbatim}
## kcca object of family 'kmedians' 
## 
## call:
## kcca(x = Info2, k = 5, family = kccaFamily("kmedians"))
## 
## cluster sizes:
## 
## 1 2 3 4 5 
## 7 3 6 3 2
\end{verbatim}

\begin{Shaded}
\begin{Highlighting}[]
\CommentTok{\#Using predict function.}
\NormalTok{clusters\_index }\OtherTok{\textless{}{-}} \FunctionTok{predict}\NormalTok{(k51)}
\FunctionTok{dist}\NormalTok{(k51}\SpecialCharTok{@}\NormalTok{centers)}
\end{Highlighting}
\end{Shaded}

\begin{verbatim}
##          1        2        3        4
## 2 2.150651                           
## 3 3.513242 4.146567                  
## 4 3.878726 4.246051 3.388339         
## 5 3.018500 3.737739 5.124420 6.043691
\end{verbatim}

\begin{Shaded}
\begin{Highlighting}[]
\FunctionTok{image}\NormalTok{(k51)}
\FunctionTok{points}\NormalTok{(Info2, }\AttributeTok{col=}\NormalTok{clusters\_index, }\AttributeTok{pch=}\DecValTok{19}\NormalTok{, }\AttributeTok{cex=}\FloatTok{0.3}\NormalTok{)}
\end{Highlighting}
\end{Shaded}

\includegraphics{Assignment4_files/figure-latex/unnamed-chunk-8-1.pdf}
b.Interpret the clusters with respect to the numerical variables used in
forming the clusters Using Kmeans method to calculate Mean.

\begin{Shaded}
\begin{Highlighting}[]
\NormalTok{Info1 }\SpecialCharTok{\%\textgreater{}\%} \FunctionTok{mutate}\NormalTok{(}\AttributeTok{Cluster =}\NormalTok{ k5}\SpecialCharTok{$}\NormalTok{cluster) }\SpecialCharTok{\%\textgreater{}\%} \FunctionTok{group\_by}\NormalTok{(Cluster) }\SpecialCharTok{\%\textgreater{}\%} \FunctionTok{summarise\_all}\NormalTok{(}\StringTok{"mean"}\NormalTok{)}
\end{Highlighting}
\end{Shaded}

\begin{verbatim}
## # A tibble: 5 x 10
##   Cluster Market_Cap  Beta PE_Ratio   ROE   ROA Asset_Turnover Leverage
##     <int>      <dbl> <dbl>    <dbl> <dbl> <dbl>          <dbl>    <dbl>
## 1       1      13.1  0.598     17.7  14.6  6.2           0.425    0.635
## 2       2      31.9  0.405     69.5  13.2  5.6           0.75     0.475
## 3       3      55.8  0.414     20.3  28.7 12.7           0.738    0.371
## 4       4       6.64 0.87      24.6  16.5  4.17          0.6      1.65 
## 5       5     157.   0.48      22.2  44.4 17.7           0.95     0.22 
## # ... with 2 more variables: Rev_Growth <dbl>, Net_Profit_Margin <dbl>
\end{verbatim}

\begin{Shaded}
\begin{Highlighting}[]
\FunctionTok{clusplot}\NormalTok{(Info2,k5}\SpecialCharTok{$}\NormalTok{cluster, }\AttributeTok{main=}\StringTok{"Clusters"}\NormalTok{,}\AttributeTok{color =} \ConstantTok{TRUE}\NormalTok{, }\AttributeTok{labels =} \DecValTok{2}\NormalTok{,}\AttributeTok{lines =} \DecValTok{0}\NormalTok{)}
\end{Highlighting}
\end{Shaded}

\includegraphics{Assignment4_files/figure-latex/unnamed-chunk-9-1.pdf}
Comapnies are categorized into different clusters as follows:

Cluster 1: ELN, MRX, WPI and AVE Cluster 2: AGN and PHA Cluster 3:
AHM,WYE,BMY,AZN, LLY, ABT, NVS and SGP Cluster 4: BAY, CHTT and IVX
Cluster 5: JNJ, MRK, PFE and GSK From the means of the cluster variables
, we can say that, Cluster 1 has the fastest revenue growth, the highest
Net Profit Margin, and the lowest PE ratio. It can be purchased or held
in reserve..Cluster 2 PE ratio is very high Cluster 3 has average risk
Cluster 4 Though it has a good PE ratio, it carries a very high risk ,
very very high leverage and low Net Profit margin , making it very risky
to own. Revenue growth is also very low.Cluster 5 has a high market
capitalization, return on investment, return on assets, asset turnover,
and net profit margin. With a low PE ratio, the stock price is
moderately valued and hence can be purchased and held evenue growth of
18.5\% is good. c.Is there a pattern in the clusters with respect to the
numerical variables (10 to 12)? (those not used informing the clusters)
\#plotting clusters against the variables to check for any patterns

\begin{Shaded}
\begin{Highlighting}[]
\NormalTok{Info3 }\OtherTok{\textless{}{-}}\NormalTok{ Info[}\DecValTok{12}\SpecialCharTok{:}\DecValTok{14}\NormalTok{] }\SpecialCharTok{\%\textgreater{}\%} \FunctionTok{mutate}\NormalTok{(}\AttributeTok{Clusters=}\NormalTok{k5}\SpecialCharTok{$}\NormalTok{cluster)}
\FunctionTok{ggplot}\NormalTok{(Info3, }\AttributeTok{mapping =} \FunctionTok{aes}\NormalTok{(}\FunctionTok{factor}\NormalTok{(Clusters), }\AttributeTok{fill =}\NormalTok{Median\_Recommendation))}\SpecialCharTok{+}\FunctionTok{geom\_bar}\NormalTok{(}\AttributeTok{position=}\StringTok{\textquotesingle{}dodge\textquotesingle{}}\NormalTok{)}\SpecialCharTok{+}\FunctionTok{labs}\NormalTok{(}\AttributeTok{x =}\StringTok{\textquotesingle{}Clusters\textquotesingle{}}\NormalTok{)}
\end{Highlighting}
\end{Shaded}

\includegraphics{Assignment4_files/figure-latex/unnamed-chunk-10-1.pdf}

\begin{Shaded}
\begin{Highlighting}[]
\FunctionTok{ggplot}\NormalTok{(Info3, }\AttributeTok{mapping =} \FunctionTok{aes}\NormalTok{(}\FunctionTok{factor}\NormalTok{(Clusters),}\AttributeTok{fill =}\NormalTok{ Location))}\SpecialCharTok{+}\FunctionTok{geom\_bar}\NormalTok{(}\AttributeTok{position =} \StringTok{\textquotesingle{}dodge\textquotesingle{}}\NormalTok{)}\SpecialCharTok{+}\FunctionTok{labs}\NormalTok{(}\AttributeTok{x =}\StringTok{\textquotesingle{}Clusters\textquotesingle{}}\NormalTok{)}
\end{Highlighting}
\end{Shaded}

\includegraphics{Assignment4_files/figure-latex/unnamed-chunk-10-2.pdf}

\begin{Shaded}
\begin{Highlighting}[]
\FunctionTok{ggplot}\NormalTok{(Info3, }\AttributeTok{mapping =} \FunctionTok{aes}\NormalTok{(}\FunctionTok{factor}\NormalTok{(Clusters),}\AttributeTok{fill =}\NormalTok{ Exchange))}\SpecialCharTok{+}\FunctionTok{geom\_bar}\NormalTok{(}\AttributeTok{position =} \StringTok{\textquotesingle{}dodge\textquotesingle{}}\NormalTok{)}\SpecialCharTok{+}\FunctionTok{labs}\NormalTok{(}\AttributeTok{x =}\StringTok{\textquotesingle{}Clusters\textquotesingle{}}\NormalTok{)}
\end{Highlighting}
\end{Shaded}

\includegraphics{Assignment4_files/figure-latex/unnamed-chunk-10-3.pdf}

--\textgreater.Clusters and the variable Median Recommendation appear to
follow a pattern. --\textgreater Except for the fact that the bulk of
the clusters/companies are listed on the NYSE and are based in the
United States, there appears to be no discernible pattern among the
clusters, locations, or exchanges. d.Provide an appropriate name for
each cluster using any or all of the variables in the dataset. Cluster
1: Best Buying Cluster 2: Highly Risky Cluster 3: Go for it Cluster 4:
Very Risky or Runaway Cluster 5: Ideal to Own

\end{document}
