% Options for packages loaded elsewhere
\PassOptionsToPackage{unicode}{hyperref}
\PassOptionsToPackage{hyphens}{url}
%
\documentclass[
]{article}
\title{FML\_Assignment\_5}
\author{Swetha}
\date{4/14/2022}

\usepackage{amsmath,amssymb}
\usepackage{lmodern}
\usepackage{iftex}
\ifPDFTeX
  \usepackage[T1]{fontenc}
  \usepackage[utf8]{inputenc}
  \usepackage{textcomp} % provide euro and other symbols
\else % if luatex or xetex
  \usepackage{unicode-math}
  \defaultfontfeatures{Scale=MatchLowercase}
  \defaultfontfeatures[\rmfamily]{Ligatures=TeX,Scale=1}
\fi
% Use upquote if available, for straight quotes in verbatim environments
\IfFileExists{upquote.sty}{\usepackage{upquote}}{}
\IfFileExists{microtype.sty}{% use microtype if available
  \usepackage[]{microtype}
  \UseMicrotypeSet[protrusion]{basicmath} % disable protrusion for tt fonts
}{}
\makeatletter
\@ifundefined{KOMAClassName}{% if non-KOMA class
  \IfFileExists{parskip.sty}{%
    \usepackage{parskip}
  }{% else
    \setlength{\parindent}{0pt}
    \setlength{\parskip}{6pt plus 2pt minus 1pt}}
}{% if KOMA class
  \KOMAoptions{parskip=half}}
\makeatother
\usepackage{xcolor}
\IfFileExists{xurl.sty}{\usepackage{xurl}}{} % add URL line breaks if available
\IfFileExists{bookmark.sty}{\usepackage{bookmark}}{\usepackage{hyperref}}
\hypersetup{
  pdftitle={FML\_Assignment\_5},
  pdfauthor={Swetha},
  hidelinks,
  pdfcreator={LaTeX via pandoc}}
\urlstyle{same} % disable monospaced font for URLs
\usepackage[margin=1in]{geometry}
\usepackage{color}
\usepackage{fancyvrb}
\newcommand{\VerbBar}{|}
\newcommand{\VERB}{\Verb[commandchars=\\\{\}]}
\DefineVerbatimEnvironment{Highlighting}{Verbatim}{commandchars=\\\{\}}
% Add ',fontsize=\small' for more characters per line
\usepackage{framed}
\definecolor{shadecolor}{RGB}{248,248,248}
\newenvironment{Shaded}{\begin{snugshade}}{\end{snugshade}}
\newcommand{\AlertTok}[1]{\textcolor[rgb]{0.94,0.16,0.16}{#1}}
\newcommand{\AnnotationTok}[1]{\textcolor[rgb]{0.56,0.35,0.01}{\textbf{\textit{#1}}}}
\newcommand{\AttributeTok}[1]{\textcolor[rgb]{0.77,0.63,0.00}{#1}}
\newcommand{\BaseNTok}[1]{\textcolor[rgb]{0.00,0.00,0.81}{#1}}
\newcommand{\BuiltInTok}[1]{#1}
\newcommand{\CharTok}[1]{\textcolor[rgb]{0.31,0.60,0.02}{#1}}
\newcommand{\CommentTok}[1]{\textcolor[rgb]{0.56,0.35,0.01}{\textit{#1}}}
\newcommand{\CommentVarTok}[1]{\textcolor[rgb]{0.56,0.35,0.01}{\textbf{\textit{#1}}}}
\newcommand{\ConstantTok}[1]{\textcolor[rgb]{0.00,0.00,0.00}{#1}}
\newcommand{\ControlFlowTok}[1]{\textcolor[rgb]{0.13,0.29,0.53}{\textbf{#1}}}
\newcommand{\DataTypeTok}[1]{\textcolor[rgb]{0.13,0.29,0.53}{#1}}
\newcommand{\DecValTok}[1]{\textcolor[rgb]{0.00,0.00,0.81}{#1}}
\newcommand{\DocumentationTok}[1]{\textcolor[rgb]{0.56,0.35,0.01}{\textbf{\textit{#1}}}}
\newcommand{\ErrorTok}[1]{\textcolor[rgb]{0.64,0.00,0.00}{\textbf{#1}}}
\newcommand{\ExtensionTok}[1]{#1}
\newcommand{\FloatTok}[1]{\textcolor[rgb]{0.00,0.00,0.81}{#1}}
\newcommand{\FunctionTok}[1]{\textcolor[rgb]{0.00,0.00,0.00}{#1}}
\newcommand{\ImportTok}[1]{#1}
\newcommand{\InformationTok}[1]{\textcolor[rgb]{0.56,0.35,0.01}{\textbf{\textit{#1}}}}
\newcommand{\KeywordTok}[1]{\textcolor[rgb]{0.13,0.29,0.53}{\textbf{#1}}}
\newcommand{\NormalTok}[1]{#1}
\newcommand{\OperatorTok}[1]{\textcolor[rgb]{0.81,0.36,0.00}{\textbf{#1}}}
\newcommand{\OtherTok}[1]{\textcolor[rgb]{0.56,0.35,0.01}{#1}}
\newcommand{\PreprocessorTok}[1]{\textcolor[rgb]{0.56,0.35,0.01}{\textit{#1}}}
\newcommand{\RegionMarkerTok}[1]{#1}
\newcommand{\SpecialCharTok}[1]{\textcolor[rgb]{0.00,0.00,0.00}{#1}}
\newcommand{\SpecialStringTok}[1]{\textcolor[rgb]{0.31,0.60,0.02}{#1}}
\newcommand{\StringTok}[1]{\textcolor[rgb]{0.31,0.60,0.02}{#1}}
\newcommand{\VariableTok}[1]{\textcolor[rgb]{0.00,0.00,0.00}{#1}}
\newcommand{\VerbatimStringTok}[1]{\textcolor[rgb]{0.31,0.60,0.02}{#1}}
\newcommand{\WarningTok}[1]{\textcolor[rgb]{0.56,0.35,0.01}{\textbf{\textit{#1}}}}
\usepackage{graphicx}
\makeatletter
\def\maxwidth{\ifdim\Gin@nat@width>\linewidth\linewidth\else\Gin@nat@width\fi}
\def\maxheight{\ifdim\Gin@nat@height>\textheight\textheight\else\Gin@nat@height\fi}
\makeatother
% Scale images if necessary, so that they will not overflow the page
% margins by default, and it is still possible to overwrite the defaults
% using explicit options in \includegraphics[width, height, ...]{}
\setkeys{Gin}{width=\maxwidth,height=\maxheight,keepaspectratio}
% Set default figure placement to htbp
\makeatletter
\def\fps@figure{htbp}
\makeatother
\setlength{\emergencystretch}{3em} % prevent overfull lines
\providecommand{\tightlist}{%
  \setlength{\itemsep}{0pt}\setlength{\parskip}{0pt}}
\setcounter{secnumdepth}{-\maxdimen} % remove section numbering
\ifLuaTeX
  \usepackage{selnolig}  % disable illegal ligatures
\fi

\begin{document}
\maketitle

\begin{Shaded}
\begin{Highlighting}[]
\CommentTok{\# installing required packages}
\FunctionTok{library}\NormalTok{(ISLR)}
\FunctionTok{library}\NormalTok{(caret)}
\end{Highlighting}
\end{Shaded}

\begin{verbatim}
## Loading required package: ggplot2
\end{verbatim}

\begin{verbatim}
## Loading required package: lattice
\end{verbatim}

\begin{Shaded}
\begin{Highlighting}[]
\FunctionTok{library}\NormalTok{(dplyr)}
\end{Highlighting}
\end{Shaded}

\begin{verbatim}
## 
## Attaching package: 'dplyr'
\end{verbatim}

\begin{verbatim}
## The following objects are masked from 'package:stats':
## 
##     filter, lag
\end{verbatim}

\begin{verbatim}
## The following objects are masked from 'package:base':
## 
##     intersect, setdiff, setequal, union
\end{verbatim}

\begin{Shaded}
\begin{Highlighting}[]
\FunctionTok{library}\NormalTok{(cluster)}
\end{Highlighting}
\end{Shaded}

\begin{verbatim}
## Warning: package 'cluster' was built under R version 4.1.3
\end{verbatim}

\begin{Shaded}
\begin{Highlighting}[]
\FunctionTok{library}\NormalTok{(factoextra)}
\end{Highlighting}
\end{Shaded}

\begin{verbatim}
## Warning: package 'factoextra' was built under R version 4.1.3
\end{verbatim}

\begin{verbatim}
## Welcome! Want to learn more? See two factoextra-related books at https://goo.gl/ve3WBa
\end{verbatim}

\begin{Shaded}
\begin{Highlighting}[]
\FunctionTok{library}\NormalTok{(NbClust)}
\FunctionTok{library}\NormalTok{(ppclust)}
\end{Highlighting}
\end{Shaded}

\begin{verbatim}
## Warning: package 'ppclust' was built under R version 4.1.3
\end{verbatim}

\begin{Shaded}
\begin{Highlighting}[]
\FunctionTok{library}\NormalTok{(dendextend)}
\end{Highlighting}
\end{Shaded}

\begin{verbatim}
## Warning: package 'dendextend' was built under R version 4.1.3
\end{verbatim}

\begin{verbatim}
## 
## ---------------------
## Welcome to dendextend version 1.15.2
## Type citation('dendextend') for how to cite the package.
## 
## Type browseVignettes(package = 'dendextend') for the package vignette.
## The github page is: https://github.com/talgalili/dendextend/
## 
## Suggestions and bug-reports can be submitted at: https://github.com/talgalili/dendextend/issues
## You may ask questions at stackoverflow, use the r and dendextend tags: 
##   https://stackoverflow.com/questions/tagged/dendextend
## 
##  To suppress this message use:  suppressPackageStartupMessages(library(dendextend))
## ---------------------
\end{verbatim}

\begin{verbatim}
## 
## Attaching package: 'dendextend'
\end{verbatim}

\begin{verbatim}
## The following object is masked from 'package:stats':
## 
##     cutree
\end{verbatim}

\begin{Shaded}
\begin{Highlighting}[]
\FunctionTok{library}\NormalTok{(tidyverse)}
\end{Highlighting}
\end{Shaded}

\begin{verbatim}
## Warning: package 'tidyverse' was built under R version 4.1.3
\end{verbatim}

\begin{verbatim}
## -- Attaching packages --------------------------------------- tidyverse 1.3.1 --
\end{verbatim}

\begin{verbatim}
## v tibble  3.1.6     v purrr   0.3.4
## v tidyr   1.2.0     v stringr 1.4.0
## v readr   2.1.2     v forcats 0.5.1
\end{verbatim}

\begin{verbatim}
## Warning: package 'readr' was built under R version 4.1.3
\end{verbatim}

\begin{verbatim}
## Warning: package 'forcats' was built under R version 4.1.3
\end{verbatim}

\begin{verbatim}
## -- Conflicts ------------------------------------------ tidyverse_conflicts() --
## x dplyr::filter() masks stats::filter()
## x dplyr::lag()    masks stats::lag()
## x purrr::lift()   masks caret::lift()
\end{verbatim}

\begin{Shaded}
\begin{Highlighting}[]
\FunctionTok{library}\NormalTok{(ggplot2)}
\FunctionTok{library}\NormalTok{(proxy)}
\end{Highlighting}
\end{Shaded}

\begin{verbatim}
## 
## Attaching package: 'proxy'
\end{verbatim}

\begin{verbatim}
## The following objects are masked from 'package:stats':
## 
##     as.dist, dist
\end{verbatim}

\begin{verbatim}
## The following object is masked from 'package:base':
## 
##     as.matrix
\end{verbatim}

\begin{Shaded}
\begin{Highlighting}[]
\CommentTok{\# to import the "cereal" data set}
\NormalTok{Cereals }\OtherTok{\textless{}{-}} \FunctionTok{read.csv}\NormalTok{(}\StringTok{"Cereals.csv"}\NormalTok{)}
\CommentTok{\#using  head to get first few rows of the data set}
\FunctionTok{head}\NormalTok{(Cereals)}
\end{Highlighting}
\end{Shaded}

\begin{verbatim}
##                        name mfr type calories protein fat sodium fiber carbo
## 1                 100%_Bran   N    C       70       4   1    130  10.0   5.0
## 2         100%_Natural_Bran   Q    C      120       3   5     15   2.0   8.0
## 3                  All-Bran   K    C       70       4   1    260   9.0   7.0
## 4 All-Bran_with_Extra_Fiber   K    C       50       4   0    140  14.0   8.0
## 5            Almond_Delight   R    C      110       2   2    200   1.0  14.0
## 6   Apple_Cinnamon_Cheerios   G    C      110       2   2    180   1.5  10.5
##   sugars potass vitamins shelf weight cups   rating
## 1      6    280       25     3      1 0.33 68.40297
## 2      8    135        0     3      1 1.00 33.98368
## 3      5    320       25     3      1 0.33 59.42551
## 4      0    330       25     3      1 0.50 93.70491
## 5      8     NA       25     3      1 0.75 34.38484
## 6     10     70       25     1      1 0.75 29.50954
\end{verbatim}

\begin{Shaded}
\begin{Highlighting}[]
\CommentTok{\#using str to analyse the structure of the data set}
\FunctionTok{str}\NormalTok{(Cereals)}
\end{Highlighting}
\end{Shaded}

\begin{verbatim}
## 'data.frame':    77 obs. of  16 variables:
##  $ name    : chr  "100%_Bran" "100%_Natural_Bran" "All-Bran" "All-Bran_with_Extra_Fiber" ...
##  $ mfr     : chr  "N" "Q" "K" "K" ...
##  $ type    : chr  "C" "C" "C" "C" ...
##  $ calories: int  70 120 70 50 110 110 110 130 90 90 ...
##  $ protein : int  4 3 4 4 2 2 2 3 2 3 ...
##  $ fat     : int  1 5 1 0 2 2 0 2 1 0 ...
##  $ sodium  : int  130 15 260 140 200 180 125 210 200 210 ...
##  $ fiber   : num  10 2 9 14 1 1.5 1 2 4 5 ...
##  $ carbo   : num  5 8 7 8 14 10.5 11 18 15 13 ...
##  $ sugars  : int  6 8 5 0 8 10 14 8 6 5 ...
##  $ potass  : int  280 135 320 330 NA 70 30 100 125 190 ...
##  $ vitamins: int  25 0 25 25 25 25 25 25 25 25 ...
##  $ shelf   : int  3 3 3 3 3 1 2 3 1 3 ...
##  $ weight  : num  1 1 1 1 1 1 1 1.33 1 1 ...
##  $ cups    : num  0.33 1 0.33 0.5 0.75 0.75 1 0.75 0.67 0.67 ...
##  $ rating  : num  68.4 34 59.4 93.7 34.4 ...
\end{verbatim}

\begin{Shaded}
\begin{Highlighting}[]
\CommentTok{\# using summary to analyse the summary of the data set}
\FunctionTok{summary}\NormalTok{(Cereals)}
\end{Highlighting}
\end{Shaded}

\begin{verbatim}
##      name               mfr                type              calories    
##  Length:77          Length:77          Length:77          Min.   : 50.0  
##  Class :character   Class :character   Class :character   1st Qu.:100.0  
##  Mode  :character   Mode  :character   Mode  :character   Median :110.0  
##                                                           Mean   :106.9  
##                                                           3rd Qu.:110.0  
##                                                           Max.   :160.0  
##                                                                          
##     protein           fat            sodium          fiber       
##  Min.   :1.000   Min.   :0.000   Min.   :  0.0   Min.   : 0.000  
##  1st Qu.:2.000   1st Qu.:0.000   1st Qu.:130.0   1st Qu.: 1.000  
##  Median :3.000   Median :1.000   Median :180.0   Median : 2.000  
##  Mean   :2.545   Mean   :1.013   Mean   :159.7   Mean   : 2.152  
##  3rd Qu.:3.000   3rd Qu.:2.000   3rd Qu.:210.0   3rd Qu.: 3.000  
##  Max.   :6.000   Max.   :5.000   Max.   :320.0   Max.   :14.000  
##                                                                  
##      carbo          sugars           potass          vitamins     
##  Min.   : 5.0   Min.   : 0.000   Min.   : 15.00   Min.   :  0.00  
##  1st Qu.:12.0   1st Qu.: 3.000   1st Qu.: 42.50   1st Qu.: 25.00  
##  Median :14.5   Median : 7.000   Median : 90.00   Median : 25.00  
##  Mean   :14.8   Mean   : 7.026   Mean   : 98.67   Mean   : 28.25  
##  3rd Qu.:17.0   3rd Qu.:11.000   3rd Qu.:120.00   3rd Qu.: 25.00  
##  Max.   :23.0   Max.   :15.000   Max.   :330.00   Max.   :100.00  
##  NA's   :1      NA's   :1        NA's   :2                        
##      shelf           weight          cups           rating     
##  Min.   :1.000   Min.   :0.50   Min.   :0.250   Min.   :18.04  
##  1st Qu.:1.000   1st Qu.:1.00   1st Qu.:0.670   1st Qu.:33.17  
##  Median :2.000   Median :1.00   Median :0.750   Median :40.40  
##  Mean   :2.208   Mean   :1.03   Mean   :0.821   Mean   :42.67  
##  3rd Qu.:3.000   3rd Qu.:1.00   3rd Qu.:1.000   3rd Qu.:50.83  
##  Max.   :3.000   Max.   :1.50   Max.   :1.500   Max.   :93.70  
## 
\end{verbatim}

Now I am scaling the data to remove NA values from the data set.

\begin{Shaded}
\begin{Highlighting}[]
\CommentTok{\# Here I am Creating duplicate of data set for preprocessing}
\NormalTok{Scaled\_Cereals }\OtherTok{\textless{}{-}}\NormalTok{ Cereals}
\CommentTok{\# Now I am scaling the data set to place it into a clustering algorithm}
\NormalTok{Scaled\_Cereals[ , }\FunctionTok{c}\NormalTok{(}\DecValTok{4}\SpecialCharTok{:}\DecValTok{16}\NormalTok{)] }\OtherTok{\textless{}{-}} \FunctionTok{scale}\NormalTok{(Cereals[ , }\FunctionTok{c}\NormalTok{(}\DecValTok{4}\SpecialCharTok{:}\DecValTok{16}\NormalTok{)])}
\CommentTok{\# Here I am Removing NA values from data set using omit function}
\NormalTok{Preprocessed\_Cereal }\OtherTok{\textless{}{-}} \FunctionTok{na.omit}\NormalTok{(Scaled\_Cereals)}
\CommentTok{\#Using head to display first few rows after removing NA}
\FunctionTok{head}\NormalTok{(Preprocessed\_Cereal)}
\end{Highlighting}
\end{Shaded}

\begin{verbatim}
##                        name mfr type   calories    protein         fat
## 1                 100%_Bran   N    C -1.8929836  1.3286071 -0.01290349
## 2         100%_Natural_Bran   Q    C  0.6732089  0.4151897  3.96137277
## 3                  All-Bran   K    C -1.8929836  1.3286071 -0.01290349
## 4 All-Bran_with_Extra_Fiber   K    C -2.9194605  1.3286071 -1.00647256
## 6   Apple_Cinnamon_Cheerios   G    C  0.1599704 -0.4982277  0.98066557
## 7               Apple_Jacks   K    C  0.1599704 -0.4982277 -1.00647256
##       sodium       fiber      carbo     sugars     potass   vitamins      shelf
## 1 -0.3539844  3.29284661 -2.5087829 -0.2343906  2.5753685 -0.1453172  0.9515734
## 2 -1.7257708 -0.06375361 -1.7409943  0.2223705  0.5160205 -1.2642598  0.9515734
## 3  1.1967306  2.87327158 -1.9969238 -0.4627711  3.1434645 -0.1453172  0.9515734
## 4 -0.2346986  4.97114672 -1.7409943 -1.6046739  3.2854885 -0.1453172  0.9515734
## 6  0.2424445 -0.27354112 -1.1011705  0.6791317 -0.4071355 -0.1453172 -1.4507595
## 7 -0.4136273 -0.48332864 -0.9732057  1.5926539 -0.9752315 -0.1453172 -0.2495930
##       weight       cups     rating
## 1 -0.1967771 -2.1100340  1.8321876
## 2 -0.1967771  0.7690100 -0.6180571
## 3 -0.1967771 -2.1100340  1.1930986
## 4 -0.1967771 -1.3795303  3.6333849
## 6 -0.1967771 -0.3052601 -0.9365625
## 7 -0.1967771  0.7690100 -0.6756899
\end{verbatim}

The total number of observations,after pre-processing and scaling the
data, went from 77 to 74. So, there were only 3 records with ``NA''
value.

\hypertarget{q-apply-hierarchical-clustering-to-the-data-using-euclidean-distance-to-the-normalized-measurements.-use-agnes-to-compare-the-clustering-from-single-linkage-complete-linkage-average-linkage-and-ward.-choose-the-best-method.}{%
\subsection{Q) Apply hierarchical clustering to the data using Euclidean
distance to the normalized measurements. Use Agnes to compare the
clustering from single linkage, complete linkage, average linkage, and
Ward. Choose the best
method.}\label{q-apply-hierarchical-clustering-to-the-data-using-euclidean-distance-to-the-normalized-measurements.-use-agnes-to-compare-the-clustering-from-single-linkage-complete-linkage-average-linkage-and-ward.-choose-the-best-method.}}

\hypertarget{single-linkage}{%
\subsection{Single Linkage:}\label{single-linkage}}

\begin{Shaded}
\begin{Highlighting}[]
\CommentTok{\# Creating the dissimilarity matrix for all the numeric values in the data set through Euclidean distance measurements}
\NormalTok{Cereal\_Euclidean }\OtherTok{\textless{}{-}} \FunctionTok{dist}\NormalTok{(Preprocessed\_Cereal[ , }\FunctionTok{c}\NormalTok{(}\DecValTok{4}\SpecialCharTok{:}\DecValTok{16}\NormalTok{)], }\AttributeTok{method =} \StringTok{"euclidean"}\NormalTok{)}
\CommentTok{\# Performing an  hierarchical clustering through the single linkage method}
\NormalTok{HC\_Single }\OtherTok{\textless{}{-}} \FunctionTok{agnes}\NormalTok{(Cereal\_Euclidean, }\AttributeTok{method =} \StringTok{"single"}\NormalTok{)}
\CommentTok{\# Here I am Plotting the results of the different methods}
\FunctionTok{plot}\NormalTok{(HC\_Single, }
     \AttributeTok{main =} \StringTok{"Customer Cereal Ratings {-} AGNES Using  Single Linkage Method"}\NormalTok{,}
     \AttributeTok{xlab =} \StringTok{"Cereal"}\NormalTok{,}
     \AttributeTok{ylab =} \StringTok{"Height"}\NormalTok{,}
     \AttributeTok{cex.axis =} \DecValTok{1}\NormalTok{,}
     \AttributeTok{cex =} \FloatTok{0.50}\NormalTok{)}
\end{Highlighting}
\end{Shaded}

\includegraphics{Assignment5_files/figure-latex/unnamed-chunk-4-1.pdf}
\includegraphics{Assignment5_files/figure-latex/unnamed-chunk-4-2.pdf}

\hypertarget{complete-linkage}{%
\section{Complete Linkage:}\label{complete-linkage}}

\begin{Shaded}
\begin{Highlighting}[]
\CommentTok{\# Performing the hierarchical clustering through the complete linkage method}
\NormalTok{HC\_Complete }\OtherTok{\textless{}{-}} \FunctionTok{agnes}\NormalTok{(Cereal\_Euclidean, }\AttributeTok{method =} \StringTok{"complete"}\NormalTok{)}
\CommentTok{\# Here I am Plotting the results of the different methods}
\FunctionTok{plot}\NormalTok{(HC\_Complete, }
     \AttributeTok{main =} \StringTok{"Customer Cereal Ratings {-} AGNES  Using Complete Linkage Method"}\NormalTok{,}
     \AttributeTok{xlab =} \StringTok{"Cereal"}\NormalTok{,}
     \AttributeTok{ylab =} \StringTok{"Height"}\NormalTok{,}
     \AttributeTok{cex.axis =} \DecValTok{1}\NormalTok{,}
     \AttributeTok{cex =} \FloatTok{0.50}\NormalTok{)}
\end{Highlighting}
\end{Shaded}

\includegraphics{Assignment5_files/figure-latex/unnamed-chunk-5-1.pdf}
\includegraphics{Assignment5_files/figure-latex/unnamed-chunk-5-2.pdf}

\hypertarget{average-linkage}{%
\section{Average Linkage:}\label{average-linkage}}

\begin{Shaded}
\begin{Highlighting}[]
\CommentTok{\# Performing the hierarchical clustering through the average linkage method}
\NormalTok{HC\_Average }\OtherTok{\textless{}{-}} \FunctionTok{agnes}\NormalTok{(Cereal\_Euclidean, }\AttributeTok{method =} \StringTok{"average"}\NormalTok{)}
\CommentTok{\# Here I am Plotting the results of the different methods}
\FunctionTok{plot}\NormalTok{(HC\_Average, }
     \AttributeTok{main =} \StringTok{"Customer Cereal Ratings {-} AGNES using Average Linkage Method"}\NormalTok{,}
     \AttributeTok{xlab =} \StringTok{"Cereal"}\NormalTok{,}
     \AttributeTok{ylab =} \StringTok{"Height"}\NormalTok{,}
     \AttributeTok{cex.axis =} \DecValTok{1}\NormalTok{,}
     \AttributeTok{cex =} \FloatTok{0.50}\NormalTok{)}
\end{Highlighting}
\end{Shaded}

\includegraphics{Assignment5_files/figure-latex/unnamed-chunk-6-1.pdf}
\includegraphics{Assignment5_files/figure-latex/unnamed-chunk-6-2.pdf}

\hypertarget{ward-method}{%
\section{Ward Method:}\label{ward-method}}

\begin{Shaded}
\begin{Highlighting}[]
\CommentTok{\# Performing the hierarchical clustering through the ward linkage method}
\NormalTok{HC\_Ward }\OtherTok{\textless{}{-}} \FunctionTok{agnes}\NormalTok{(Cereal\_Euclidean, }\AttributeTok{method =} \StringTok{"ward"}\NormalTok{)}
\CommentTok{\#Here I am  Plotting the results of the different methods}
\FunctionTok{plot}\NormalTok{(HC\_Ward, }
     \AttributeTok{main =} \StringTok{"Customer Cereal Ratings {-} AGNES using Ward Linkage Method"}\NormalTok{,}
     \AttributeTok{xlab =} \StringTok{"Cereal"}\NormalTok{,}
     \AttributeTok{ylab =} \StringTok{"Height"}\NormalTok{,}
     \AttributeTok{cex.axis =} \DecValTok{1}\NormalTok{,}
     \AttributeTok{cex =} \FloatTok{0.55}\NormalTok{)}
\end{Highlighting}
\end{Shaded}

\includegraphics{Assignment5_files/figure-latex/unnamed-chunk-7-1.pdf}
\includegraphics{Assignment5_files/figure-latex/unnamed-chunk-7-2.pdf}
The clustering structure is closer if the value is close to 1.0. As a
result, the method with the closest value to 1.0 will be chosen. Single
Linkage: 0.61 Complete Linkage: 0.84 Average Linkage: 0.78 Ward Method:
0.90 Here From the result, The best clustering model is the Ward method.

\hypertarget{q-how-many-clusters-would-you-choose}{%
\subsection{Q) How many clusters would you
choose?}\label{q-how-many-clusters-would-you-choose}}

\hypertarget{here-i-am-using-elbow-and-silhouette-methods-to-determine-the-appropriate-number-of-clusters.}{%
\section{Here I am using elbow and silhouette methods to determine the
appropriate number of
clusters.}\label{here-i-am-using-elbow-and-silhouette-methods-to-determine-the-appropriate-number-of-clusters.}}

\hypertarget{elbow-method}{%
\subsection{Elbow Method:}\label{elbow-method}}

\begin{Shaded}
\begin{Highlighting}[]
\FunctionTok{fviz\_nbclust}\NormalTok{(Preprocessed\_Cereal[ , }\FunctionTok{c}\NormalTok{(}\DecValTok{4}\SpecialCharTok{:}\DecValTok{16}\NormalTok{)], hcut, }\AttributeTok{method =} \StringTok{"wss"}\NormalTok{, }\AttributeTok{k.max =} \DecValTok{25}\NormalTok{) }\SpecialCharTok{+}
  \FunctionTok{labs}\NormalTok{(}\AttributeTok{title =} \StringTok{"Optimal Number of Clusters using Elbow Method"}\NormalTok{) }\SpecialCharTok{+}
  \FunctionTok{geom\_vline}\NormalTok{(}\AttributeTok{xintercept =} \DecValTok{12}\NormalTok{, }\AttributeTok{linetype =} \DecValTok{2}\NormalTok{)}
\end{Highlighting}
\end{Shaded}

\includegraphics{Assignment5_files/figure-latex/unnamed-chunk-8-1.pdf}

\#\#Silhouette Method:

\begin{Shaded}
\begin{Highlighting}[]
\FunctionTok{fviz\_nbclust}\NormalTok{(Preprocessed\_Cereal[ , }\FunctionTok{c}\NormalTok{(}\DecValTok{4}\SpecialCharTok{:}\DecValTok{16}\NormalTok{)], }
\NormalTok{                               hcut, }
                               \AttributeTok{method =} \StringTok{"silhouette"}\NormalTok{, }
                               \AttributeTok{k.max =} \DecValTok{25}\NormalTok{) }\SpecialCharTok{+}
  \FunctionTok{labs}\NormalTok{(}\AttributeTok{title =} \StringTok{"Optimal Number of Clusters using Silhouette Method"}\NormalTok{)}
\end{Highlighting}
\end{Shaded}

\includegraphics{Assignment5_files/figure-latex/unnamed-chunk-9-1.pdf}
Here from the results of the elbow and silhouette methods,we can see
that the optimal number of clusters would be 12.

\begin{Shaded}
\begin{Highlighting}[]
\CommentTok{\#Here I am  Plotting the Ward hierarchical tree with the 12 clusters outlined for reference}
\FunctionTok{plot}\NormalTok{(HC\_Ward, }
     \AttributeTok{main =} \StringTok{"AGNES {-} Ward Linkage Method using 12 Clusters Outlined"}\NormalTok{,}
     \AttributeTok{xlab =} \StringTok{"Cereal"}\NormalTok{,}
     \AttributeTok{ylab =} \StringTok{"Height"}\NormalTok{,}
     \AttributeTok{cex.axis =} \DecValTok{1}\NormalTok{,}
     \AttributeTok{cex =} \FloatTok{0.50}\NormalTok{,)}
\end{Highlighting}
\end{Shaded}

\includegraphics{Assignment5_files/figure-latex/unnamed-chunk-10-1.pdf}

\begin{Shaded}
\begin{Highlighting}[]
\FunctionTok{rect.hclust}\NormalTok{(HC\_Ward, }\AttributeTok{k =} \DecValTok{12}\NormalTok{, }\AttributeTok{border =} \DecValTok{1}\SpecialCharTok{:}\DecValTok{12}\NormalTok{)}
\end{Highlighting}
\end{Shaded}

\includegraphics{Assignment5_files/figure-latex/unnamed-chunk-10-2.pdf}

\hypertarget{q-the-elementary-public-schools-would-like-to-choose-a-set-of-cereals-to-include-in-their-daily-cafeterias.-every-day-a-different-cereal-is-offered-but-all-cereals-should-support-a-healthy-diet.-for-this-goal-you-are-requested-to-find-a-cluster-of-healthy-cereals.-should-the-data-be-normalized-if-not-how-should-they-be-used-in-the-cluster-analysis}{%
\subsection{Q) The elementary public schools would like to choose a set
of Cereals to include in their daily cafeterias. Every day a different
cereal is offered, but all Cereals should support a healthy diet. For
this goal, you are requested to find a cluster of ``healthy Cereals.''
Should the data be normalized? If not, how should they be used in the
cluster
analysis?}\label{q-the-elementary-public-schools-would-like-to-choose-a-set-of-cereals-to-include-in-their-daily-cafeterias.-every-day-a-different-cereal-is-offered-but-all-cereals-should-support-a-healthy-diet.-for-this-goal-you-are-requested-to-find-a-cluster-of-healthy-cereals.-should-the-data-be-normalized-if-not-how-should-they-be-used-in-the-cluster-analysis}}

Since the nutritional information for cereal is normalized based on the
sample of cereal being assessed, normalizing the data would not be
appropriate in this circumstance. As a result, the data collected could
only include Cereals with a very high sugar content and very little
fiber, iron, or other nutritional information. Once the cereal has been
normalized throughout the sample set, it's impossible to say how much
nutrition it will supply a child. We might assume that a cereal with an
iron level of 0.999 includes almost all of the nutrional iron that a
child requires; nevertheless, it could simply be the best of the worst
in the sample set with no nutrional value. As a result, converting the
data to a ratio of daily suggested calories, fiber, carbs, and other
nutrients for a child would be a better approach to preprocess it. This
would allow analysts to make more educated cluster judgments during the
review process while also preventing a few larger variables from
overriding the distance estimations. When examining the clusters, an
analyst may examine the cluster average to determine what percentage of
a student's daily nutritional needs would be met by XX cereal. Employees
would be able to make well-informed decisions on which ``healthy''
cereal clusters to choose as a result of this.

\end{document}
